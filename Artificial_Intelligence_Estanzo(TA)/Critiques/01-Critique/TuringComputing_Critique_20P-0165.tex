\documentclass{article}

% Language setting
% Replace `english' with e.g. `spanish' to change the document language
\usepackage[english]{babel}

% Set page size and margins
% Replace `letterpaper' with `a4paper' for UK/EU standard size
\usepackage[letterpaper,top=2cm,bottom=2cm,left=3cm,right=3cm,marginparwidth=1.75cm]{geometry}

% Useful packages
\usepackage{amsmath}
\usepackage{graphicx}
\usepackage[colorlinks=true, allcolors=blue]{hyperref}

\title{Computing Machinery and Intelligence}
\author{Jawad Ahmed(20P-0165) \ Section: BCS-6A}

\begin{document}
\maketitle

\section{Summary of Computing Machinery and Intelligence}
The paper is about 'Can Machine Think' and Alan M.Turing argues that there is no convincing argument that machines cannot think intelligently like humans and we can take different approaches to enhance machine intelligence. He describes the question 'Can Machine think?' in terms of a game called 'Imitation game' where an interrogator try to distinguish between human and a machine based on their responses. Alan M.Turing discusses different arguments about whether machine can think and improve itself or not.He hopes that machines will eventually compete with men in all pure intellectual fields.


\section{Opinion/Weaknesses/Strengths}
The way Alan M.Turing has explained about the intelligence of machines and shows that machines can be as intelligent as humans has helped researches to create intelligent machines. All the work we are seeing in the field of Machine Learning, Artificial Intelligence is possible due Alan M.Turing work. At that time there are not such high computational power machines but still his working at time helping us in the future for creating intelligent machines.

The test the Alan  M.Turing used to measure the intelligence of machines is not enough. The test is only checking whether the machines can mimic the human or not. For example: Generally humans are slow in doing mathematics if we place a well trained machine in mathematics there machine may not mimic the human but machine is still intelligent. 

The paper completely explains about the intellgence of machines and whether machines can be as intelligent as humans. This paper opens new gateways for future research and experiments.


\section{Comment on how the current approach can be improved}
There should be more focus on making better machines to solve specific problems rather than making machines to mimic humans. Second there should be different tests to check that the machine is actually exhibiting intelligent behaviour.Third there should be taken into account different types of intelligence, each requiring different capabilities and skills. Lastly there would be a better way to check whether the machine is exhibiting the intelligent behaviour or not rather than only checking with the help of 'Imitation Game' that is not enough to check the machine intelligence.

\section{Questions}
\begin{enumerate}
  \item What would happen if that machine is not good with the question asked by the judge but it is intelligent/best in some other tasks? Would that make machine unintelligent if it loses the 'Imitation Game'?
  \item By only winning the 'Imitation Game' will make the machine intelligent like human?
  \item Is it possible to measure machine intelligence by not including many other aspects, such as perception, creativity, emotion and consciousness?
\end{enumerate}
\end{document}