\documentclass{article}

% Language setting
% Replace `english' with e.g. `spanish' to change the document language
\usepackage[english]{babel}

% Set page size and margins
% Replace `letterpaper' with `a4paper' for UK/EU standard size
\usepackage[letterpaper,top=2cm,bottom=2cm,left=3cm,right=3cm,marginparwidth=1.75cm]{geometry}

% Useful packages
\usepackage{amsmath}
\usepackage{graphicx}
\usepackage[colorlinks=true, allcolors=blue]{hyperref}

\title{Report For Assignment Two}
\author{Jawad Ahmed(20P-0165) \ Section: BCS-6A}

\begin{document}
\maketitle

\section{Introduction}

The purpose of the assignment is to create a frequency analysis model that will generate Cipher which will be used to decrypt the encrypted message.

\section{Explanation of Assignment}

\subsection{Train Model Function}

The train model method is responsible for reading a directory of plain text files and generating a frequency dictionary of the characters in those files.

To do this, it creates an empty dictionary rdict with ASCII codes of characters ranging from 32 to 126 (which includes all printable ASCII characters). It also includes a special key for the newline character with ASCII code 10.

For each file in the directory, it opens the file and reads each line character by character. It then gets the ASCII code of each character and checks if it falls within the range of printable ASCII characters. If it does, it increments the count of that character in the rdict dictionary. If the character is a newline character, it increments the count for the newline key in the dictionary.

Finally, it returns the rdict dictionary which contains the frequencies of all printable ASCII characters and the newline character in the given set of files.

I have trained the my model on almost 250 books 100+ articles. 

\subsection{Generate Cipher Function}
The generatecipher method takes an input file and the frequency dictionary as arguments. It first computes the frequency dictionary of the input file similar to the trainmodel method. It then compares the frequency dictionary of the input file with the frequency dictionary generated from the training data to generate a cipher dictionary. The cipher dictionary maps the ASCII code of each character in the input file to a new ASCII code based on its frequency in the training data. It then stores the cipher dictionary in a file and generates the plain text for the input file using the cipher dictionary. Finally, it applies the WordNet algorithm to correct any errors in the plain text.

\subsection{Store Cipher File Function}
The storecipherfile method writes the cipher dictionary to a file. The generateplaintext method generates plain text for the input file by reading each character from the input file, checking if the character is present in the cipher dictionary, and writing the corresponding key for the value in the cipher dictionary to the output file. If the character is not present in the cipher dictionary, it writes the character as it is to the output file.

\subsection{Apply Word Net Function}
The applywordnet method applies the WordNet algorithm to correct any errors in the plain text generated by the generateplaintext method. It first reads the output file generated by the generateplaintext method, splits the text into words, and checks if each word is present in the WordNet dictionary. If a word is not present in the dictionary, it finds the most similar word in the dictionary and replaces it in the text. Finally, it writes the corrected text to the output file.


\subsection{Compare Dictionaries Function}
The comparedictionaries method takes two frequency dictionaries as input and generates a cipher dictionary by mapping the ASCII codes of each character in the first dictionary to a new ASCII code based on its frequency in the second dictionary. The resulting cipher dictionary maps ASCII codes from the first dictionary to ASCII codes in the second dictionary.


\end{document}